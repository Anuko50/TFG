\documentclass[a4paper,10pt]{article}
\usepackage[utf8]{inputenc}
\usepackage[spanish,es-tabla]{babel}
\usepackage[top=3cm, bottom=2cm, right=1.5cm, left=3cm]{geometry}
\usepackage[usenames,dvipsnames,svgnames,table]{xcolor}

\title{Tablón Alterado}
\author{$a^nb^nc^n-MT$}
\date{}

\begin{document}
\maketitle

\section{Tabla alterada}

Aquí mostraremos la tabla alterada de manera aleatoria. Lo que se puede encontrar es que:
\begin{enumerate}
\item Se haya introducido un carácter que no pertenece al conjunto C.
\item Se haya cambiado un carácter a otro perteneciente al conjunto C.
\item Se haya cambiado la primera fila por otra aleatoria del tablón.
\item Se haya añadido un estado de más.
\item Eliminar estados para quitar el sentido del tablón, cambiándolos por otro signo.
\end{enumerate}\begin{table}[h]
\centering
\begin{tabular}{|l|l|l|l|l|l|l|l|l|l|l|l|l|}
\hline
	\#  &   B   &   X   &   $q_0$  &   X   &   X   &   B   &   B   &   B   &   B   &   B   &   B   &   \#	\\ \hline
	\#  &   X   &   $q_1$  &   b   &   c   &   B   &   B   &   B   &   B   &   B   &   B   &   B   &   \#	\\ \hline
	\#  &   X   &   X   &   $q_2$  &   c   &   B   &   B   &   B   &   B   &   B   &   B   &   B   &   \#	\\ \hline
	\#  &   X   &   $q_3$  &   X   &   X   &   B   &   B   &   B   &   B   &   B   &   B   &   B   &   \#	\\ \hline
	\#  &   $q_3$  &   X   &   X   &   X   &   B   &   B   &   B   &   B   &   B   &   B   &   B   &   \#	\\ \hline
	\#  &   $q_3$  &   B   &   X   &   X   &   X   &   B   &   B   &   B   &   B   &   B   &   B   &   \#	\\ \hline
	\#  &   B   &   $q_0$  &   X   &   X   &   X   &   B   &   B   &   B   &   B   &   B   &   B   &   \#	\\ \hline
	\#  &   $q_0$  &   a   &   b   &   c   &   B   &   B   &   B   &   B   &   B   &   B   &   B   &   \#	\\ \hline
	\#  &   B   &   a   &   X   &   $q_0$  &   X   &   B   &   B   &   B   &   B   &   B   &   B   &   \#	\\ \hline
	\#  &   B   &   X   &   X   &   X   &   $q_0$  &   B   &   B   &   B   &   B   &   B   &   B   &   \#	\\ \hline
	\#  &   B   &   X   &   X   &   X   &   B   &   $q_6$  &   B   &   B   &   B   &   B   &   B   &   \#	\\ \hline
	\#  &   $q_0$  &   a   &   b   &   c   &   B   &   B   &   B   &   B   &   B   &   B   &   B   &   \#	\\ \hline
	\#  &   B   &   X   &   X   &   X   &   B   &   $q_6$  &   B   &   B   &   B   &   B   &   B   &   \#	\\ \hline
\end{tabular}
\end{table}
\section{Ventanas ilegales}
En este apartado se mostrarán las ventanas ilegales que nacen del tablón alterado.\newline\begin{table}[h!]
\centering
\begin{tabular}{|l|l|l|}
\hline
	\#  &   B   &   X	\\ \hline
	\#  &   X   &   $q_1$	\\ \hline
\end{tabular}
\end{table}

Se trata de la ventana cuya casilla central superior es la celda de la fila 1 y columna 2\newline
La ventana es ilegal porque, aunque aparentemente pueda parecer legal, no se ha podido llegar a ella desde ninguna transición.\newline
\begin{table}[h!]
\centering
\begin{tabular}{|l|l|l|}
\hline
	B   &   X   &   $q_0$	\\ \hline
	X   &   $q_1$  &   b	\\ \hline
\end{tabular}
\end{table}

Se trata de la ventana cuya casilla central superior es la celda de la fila 1 y columna 3\newline
La ventana es ilegal porque, aunque aparentemente pueda parecer legal, no se ha podido llegar a ella desde ninguna transición.\newline
\begin{table}[h!]
\centering
\begin{tabular}{|l|l|l|}
\hline
	X   &   $q_0$  &   X	\\ \hline
	$q_1$  &   b   &   c	\\ \hline
\end{tabular}
\end{table}

Se trata de la ventana cuya casilla central superior es la celda de la fila 1 y columna 4\newline
La ventana es ilegal porque, aunque aparentemente pueda parecer legal, no se ha podido llegar a ella desde ninguna transición.\newline
\begin{table}[h!]
\centering
\begin{tabular}{|l|l|l|}
\hline
	$q_0$  &   X   &   X	\\ \hline
	b   &   c   &   B	\\ \hline
\end{tabular}
\end{table}

Se trata de la ventana cuya casilla central superior es la celda de la fila 1 y columna 5\newline
La ventana es ilegal porque, aunque aparentemente pueda parecer legal, no se ha podido llegar a ella desde ninguna transición.\newline
\begin{table}[h!]
\centering
\begin{tabular}{|l|l|l|}
\hline
	X   &   X   &   B	\\ \hline
	c   &   B   &   B	\\ \hline
\end{tabular}
\end{table}

Se trata de la ventana cuya casilla central superior es la celda de la fila 1 y columna 6\newline
La ventana es ilegal porque, aunque aparentemente pueda parecer legal, no se ha podido llegar a ella desde ninguna transición.\newline
\begin{table}[h!]
\centering
\begin{tabular}{|l|l|l|}
\hline
	X   &   B   &   B	\\ \hline
	B   &   B   &   B	\\ \hline
\end{tabular}
\end{table}

Se trata de la ventana cuya casilla central superior es la celda de la fila 1 y columna 7\newline
La ventana es ilegal porque, aunque aparentemente pueda parecer legal, no se ha podido llegar a ella desde ninguna transición.\newline
\begin{table}[h!]
\centering
\begin{tabular}{|l|l|l|}
\hline
	X   &   X   &   B	\\ \hline
	X   &   X   &   X	\\ \hline
\end{tabular}
\end{table}

Se trata de la ventana cuya casilla central superior es la celda de la fila 5 y columna 5\newline
La ventana es ilegal porque, aunque aparentemente pueda parecer legal, no se ha podido llegar a ella desde ninguna transición.\newline
\begin{table}[h!]
\centering
\begin{tabular}{|l|l|l|}
\hline
	X   &   B   &   B	\\ \hline
	X   &   X   &   B	\\ \hline
\end{tabular}
\end{table}

Se trata de la ventana cuya casilla central superior es la celda de la fila 5 y columna 6\newline
La ventana es ilegal porque, aunque aparentemente pueda parecer legal, no se ha podido llegar a ella desde ninguna transición.\newline
\begin{table}[h!]
\centering
\begin{tabular}{|l|l|l|}
\hline
	B   &   B   &   B	\\ \hline
	X   &   B   &   B	\\ \hline
\end{tabular}
\end{table}

Se trata de la ventana cuya casilla central superior es la celda de la fila 5 y columna 7\newline
La ventana es ilegal porque, aunque aparentemente pueda parecer legal, no se ha podido llegar a ella desde ninguna transición.\newline
\begin{table}[h!]
\centering
\begin{tabular}{|l|l|l|}
\hline
	\#  &   B   &   $q_0$	\\ \hline
	\#  &   $q_0$  &   a	\\ \hline
\end{tabular}
\end{table}

Se trata de la ventana cuya casilla central superior es la celda de la fila 7 y columna 2\newline
La ventana es ilegal porque, aunque aparentemente pueda parecer legal, no se ha podido llegar a ella desde ninguna transición.\newline
\begin{table}[h!]
\centering
\begin{tabular}{|l|l|l|}
\hline
	B   &   $q_0$  &   X	\\ \hline
	$q_0$  &   a   &   b	\\ \hline
\end{tabular}
\end{table}

Se trata de la ventana cuya casilla central superior es la celda de la fila 7 y columna 3\newline
La ventana es ilegal porque, aunque aparentemente pueda parecer legal, no se ha podido llegar a ella desde ninguna transición.\newline
\begin{table}[h!]
\centering
\begin{tabular}{|l|l|l|}
\hline
	$q_0$  &   X   &   X	\\ \hline
	a   &   b   &   c	\\ \hline
\end{tabular}
\end{table}

Se trata de la ventana cuya casilla central superior es la celda de la fila 7 y columna 4\newline
La ventana es ilegal porque, aunque aparentemente pueda parecer legal, no se ha podido llegar a ella desde ninguna transición.\newline
\begin{table}[h!]
\centering
\begin{tabular}{|l|l|l|}
\hline
	X   &   X   &   X	\\ \hline
	b   &   c   &   B	\\ \hline
\end{tabular}
\end{table}

Se trata de la ventana cuya casilla central superior es la celda de la fila 7 y columna 5\newline
La ventana es ilegal porque, aunque aparentemente pueda parecer legal, no se ha podido llegar a ella desde ninguna transición.\newline
\begin{table}[h!]
\centering
\begin{tabular}{|l|l|l|}
\hline
	X   &   X   &   B	\\ \hline
	c   &   B   &   B	\\ \hline
\end{tabular}
\end{table}

Se trata de la ventana cuya casilla central superior es la celda de la fila 7 y columna 6\newline
La ventana es ilegal porque, aunque aparentemente pueda parecer legal, no se ha podido llegar a ella desde ninguna transición.\newline
\begin{table}[h!]
\centering
\begin{tabular}{|l|l|l|}
\hline
	X   &   B   &   B	\\ \hline
	B   &   B   &   B	\\ \hline
\end{tabular}
\end{table}

Se trata de la ventana cuya casilla central superior es la celda de la fila 7 y columna 7\newline
La ventana es ilegal porque, aunque aparentemente pueda parecer legal, no se ha podido llegar a ella desde ninguna transición.\newline
\begin{table}[h!]
\centering
\begin{tabular}{|l|l|l|}
\hline
	\#  &   $q_0$  &   a	\\ \hline
	\#  &   B   &   a	\\ \hline
\end{tabular}
\end{table}

Se trata de la ventana cuya casilla central superior es la celda de la fila 8 y columna 2\newline
La ventana es ilegal porque hay un estado en la celda principal de la primera fila y no hay estado en la segunda.\newline
\begin{table}[h!]
\centering
\begin{tabular}{|l|l|l|}
\hline
	$q_0$  &   a   &   b	\\ \hline
	B   &   a   &   X	\\ \hline
\end{tabular}
\end{table}

Se trata de la ventana cuya casilla central superior es la celda de la fila 8 y columna 3\newline
La ventana es ilegal porque, aunque aparentemente pueda parecer legal, no se ha podido llegar a ella desde ninguna transición.\newline
\begin{table}[h!]
\centering
\begin{tabular}{|l|l|l|}
\hline
	a   &   b   &   c	\\ \hline
	a   &   X   &   $q_0$	\\ \hline
\end{tabular}
\end{table}

Se trata de la ventana cuya casilla central superior es la celda de la fila 8 y columna 4\newline
La ventana es ilegal porque, aunque aparentemente pueda parecer legal, no se ha podido llegar a ella desde ninguna transición.\newline
\begin{table}[h!]
\centering
\begin{tabular}{|l|l|l|}
\hline
	b   &   c   &   B	\\ \hline
	X   &   $q_0$  &   X	\\ \hline
\end{tabular}
\end{table}

Se trata de la ventana cuya casilla central superior es la celda de la fila 8 y columna 5\newline
La ventana es ilegal porque hay un estado en la celda principal de la segunda fila y no hay estado en la primera.\newline
\begin{table}[h!]
\centering
\begin{tabular}{|l|l|l|}
\hline
	c   &   B   &   B	\\ \hline
	$q_0$  &   X   &   B	\\ \hline
\end{tabular}
\end{table}

Se trata de la ventana cuya casilla central superior es la celda de la fila 8 y columna 6\newline
La ventana es ilegal porque, aunque aparentemente pueda parecer legal, no se ha podido llegar a ella desde ninguna transición.\newline
\begin{table}[h!]
\centering
\begin{tabular}{|l|l|l|}
\hline
	B   &   B   &   B	\\ \hline
	X   &   B   &   B	\\ \hline
\end{tabular}
\end{table}

Se trata de la ventana cuya casilla central superior es la celda de la fila 8 y columna 7\newline
La ventana es ilegal porque, aunque aparentemente pueda parecer legal, no se ha podido llegar a ella desde ninguna transición.\newline
\begin{table}[h!]
\centering
\begin{tabular}{|l|l|l|}
\hline
	\#  &   B   &   a	\\ \hline
	\#  &   B   &   X	\\ \hline
\end{tabular}
\end{table}

Se trata de la ventana cuya casilla central superior es la celda de la fila 9 y columna 2\newline
La ventana es ilegal porque, aunque aparentemente pueda parecer legal, no se ha podido llegar a ella desde ninguna transición.\newline
\begin{table}[h!]
\centering
\begin{tabular}{|l|l|l|}
\hline
	B   &   a   &   X	\\ \hline
	B   &   X   &   X	\\ \hline
\end{tabular}
\end{table}

Se trata de la ventana cuya casilla central superior es la celda de la fila 9 y columna 3\newline
La ventana es ilegal porque, aunque aparentemente pueda parecer legal, no se ha podido llegar a ella desde ninguna transición.\newline
\begin{table}[h!]
\centering
\begin{tabular}{|l|l|l|}
\hline
	a   &   X   &   $q_0$	\\ \hline
	X   &   X   &   X	\\ \hline
\end{tabular}
\end{table}

Se trata de la ventana cuya casilla central superior es la celda de la fila 9 y columna 4\newline
La ventana es ilegal porque, aunque aparentemente pueda parecer legal, no se ha podido llegar a ella desde ninguna transición.\newline
\begin{table}[h!]
\centering
\begin{tabular}{|l|l|l|}
\hline
	\#  &   B   &   X	\\ \hline
	\#  &   $q_0$  &   a	\\ \hline
\end{tabular}
\end{table}

Se trata de la ventana cuya casilla central superior es la celda de la fila 11 y columna 2\newline
La ventana es ilegal porque hay un estado en la celda principal de la segunda fila y no hay estado en la primera.\newline
\begin{table}[h!]
\centering
\begin{tabular}{|l|l|l|}
\hline
	B   &   $q_6$  &   B	\\ \hline
	B   &   B   &   B	\\ \hline
\end{tabular}
\end{table}

Se trata de la ventana cuya casilla central superior es la celda de la fila 11 y columna 7\newline
La ventana es ilegal porque hay un estado en la celda principal de la primera fila y no hay estado en la segunda.\newline
\begin{table}[h!]
\centering
\begin{tabular}{|l|l|l|}
\hline
	\#  &   $q_0$  &   a	\\ \hline
	\#  &   B   &   X	\\ \hline
\end{tabular}
\end{table}

Se trata de la ventana cuya casilla central superior es la celda de la fila 12 y columna 2\newline
La ventana es ilegal porque hay un estado en la celda principal de la primera fila y no hay estado en la segunda.\newline
\begin{table}[h!]
\centering
\begin{tabular}{|l|l|l|}
\hline
	$q_0$  &   a   &   b	\\ \hline
	B   &   X   &   X	\\ \hline
\end{tabular}
\end{table}

Se trata de la ventana cuya casilla central superior es la celda de la fila 12 y columna 3\newline
La ventana es ilegal porque, aunque aparentemente pueda parecer legal, no se ha podido llegar a ella desde ninguna transición.\newline
\begin{table}[h!]
\centering
\begin{tabular}{|l|l|l|}
\hline
	a   &   b   &   c	\\ \hline
	X   &   X   &   X	\\ \hline
\end{tabular}
\end{table}

Se trata de la ventana cuya casilla central superior es la celda de la fila 12 y columna 4\newline
La ventana es ilegal porque, aunque aparentemente pueda parecer legal, no se ha podido llegar a ella desde ninguna transición.\newline
\begin{table}[h!]
\centering
\begin{tabular}{|l|l|l|}
\hline
	b   &   c   &   B	\\ \hline
	X   &   X   &   B	\\ \hline
\end{tabular}
\end{table}

Se trata de la ventana cuya casilla central superior es la celda de la fila 12 y columna 5\newline
La ventana es ilegal porque, aunque aparentemente pueda parecer legal, no se ha podido llegar a ella desde ninguna transición.\newline
\begin{table}[h!]
\centering
\begin{tabular}{|l|l|l|}
\hline
	c   &   B   &   B	\\ \hline
	X   &   B   &   $q_6$	\\ \hline
\end{tabular}
\end{table}

Se trata de la ventana cuya casilla central superior es la celda de la fila 12 y columna 6\newline
La ventana es ilegal porque, aunque aparentemente pueda parecer legal, no se ha podido llegar a ella desde ninguna transición.\newline
\begin{table}[h!]
\centering
\begin{tabular}{|l|l|l|}
\hline
	B   &   B   &   B	\\ \hline
	B   &   $q_6$  &   B	\\ \hline
\end{tabular}
\end{table}

Se trata de la ventana cuya casilla central superior es la celda de la fila 12 y columna 7\newline
La ventana es ilegal porque hay un estado en la celda principal de la segunda fila y no hay estado en la primera.\newline
\begin{table}[h!]
\centering
\begin{tabular}{|l|l|l|}
\hline
	B   &   B   &   B	\\ \hline
	$q_6$  &   B   &   B	\\ \hline
\end{tabular}
\end{table}

Se trata de la ventana cuya casilla central superior es la celda de la fila 12 y columna 8\newline
La ventana es ilegal porque, aunque aparentemente pueda parecer legal, no se ha podido llegar a ella desde ninguna transición.\newline
\end{document}
